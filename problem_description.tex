\noindent
\begin{tabular}{@{}p{4cm}l}
\textbf{Title:} 	& \thetitle \\
\textbf{Student:}	& \theauthor \\
\end{tabular}

\vspace{4ex}
\noindent\textbf{Problem description:}


Functional encryption is a new generalization of public key cryptography which allows very flexible access
control to encrypted data. It is natural to consider extensions of the functional idea to other cryptographic
primitives. One such primitive is key exchange. 
Since the scheme provides a much more dynamic and flexible way of exchanging secrets, it is natural to consider systems where users join and leave dynamically - such as chat rooms, Internet forums or similar distributed systems. 

This project will explain functional key exchange and compare the technique to traditional group key exchange mechanisms, considering both efficiency and security properties. It will contain a prototype implementation of an attribute based authenticated key exchange scheme in a distributed environment, based on existing work done on attribute based key exchange - using the Charm framework and Python. Different application areas for such a system will be explored and problems arising discussed.

\noindent
\begin{tabular}{@{}p{4cm}l}
\textbf{Responsible professor:} 	& \theprofessor \\
\textbf{Supervisor:}			& \thesupervisor \\
\end{tabular}