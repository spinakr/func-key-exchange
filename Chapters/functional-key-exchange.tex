\chapter{Functional Key Exchange}\label{chp:funckeyenc} 
When communicating on the Internet it is important to control what entities have access to the messages. In most cases it is important that the users can trust that the their communication cannot be stolen or eavesdropped on. Encryption is used to secure communication, to do this efficiently a shared key is usually needed. Functional key exchange is in our context defined as a set of key exchange mechanisms using some function to decided if a participant should be allowed to take part in, or be allowed access to, the key exchange. The functions will use some arguments as input and based on these decide if the session key should be output or not. This chapter will explain some proposed schemes trying to adapt this idea, then further explore possibly useful application areas and ideas. \Gls{ibake} and \gls{abake} are both examples of functional key exchange, with the latter being a generalization of the former. Since \gls{abake} will be used in the implementation later, this chapter will only introduce the basic ideas and principles with the implementation providing a more in depth description.

\section{Identity-based Authenticated Key Exchange}
\Gls{ibe} as described in \ref{subsec:IBE}, can by utilised to provide two-party mutually \gls{ake} \cite{ibake}. The approach is based on a Diffie-Hellman key exchange using an elliptic curve. Each party chose random points $a,b$. $a^p, b^p$ are then encrypted using the other parties public key and then exchanged in succession. B will include $p^a$ which he recieved from a, this is done so that A can verify that B actually was able to decrypt what he sent. B actually adds to what he receives from A by decrypting and adding his contribution and then encrypting again. After decrypting, the session key is the product $a^{bp}$, which both can calculate. After exchanging secrets, A has to authenticate himself in the same way as B did, by sending the secret he got from B back, to show that he was able to decrypt what B sent. This technique provides mutual implicit authentication between the participants, since only the users with the correct identity can decrypt. Both parties can thus be sure that no other user than the one possessing the private key corresponding to the identity, can produce the session key. Protocol \ref{protocol:ibake} shows the procedure as described by Kolesnikov et al. \cite{ibake}. 

\newtheorem{protocol}{Protocol} %%move deffenition to main
\begin{protocol}\label{protocol:ibake}

\[
\begin{array}{@{}l@{}c@{}l@{}}
\toprule
\text{A - given curve and point p} && \text{B - given curve and point p} \\
\toprule
\text{chose random point a} \\
& \xrightarrow{\textstyle IBEnc_B(p^a)} \\
& & \text{chose random point b}\\
& \xleftarrow{\textstyle IBEnc_B(p^a, p^b)} \\
\pbox{20cm}{verify $p^a$ after decrypting\\ using private key}\\
& \xrightarrow{\textstyle IBEnc_B(p^b)}\\
& & \pbox{20cm}{verify $p^b$ to authenticate that A\\ actually decrypted the message}\\ 
& \pbox{20cm}{sk = $p^{ab}$\\sid=$(p^a, p^b)$}\\
\bottomrule
\end{array}
\]
\end{protocol}

This implementation demonstrates a scheme for key exchange between two parties with the focus on assuring authenticity of the identities of the participants. This is mostly a more effective implementation of public key crypto systems, the main difference from previously popular systems is the removal of \gls{pki} by switching from \glspl{ca} to \glspl{kms}. Point being that the main idea is still to encrypt some message or symmetric key for \emph{one} specific user. Another point in favor of \gls{ibake} is that it may make encryption using public key crypto

\section{Attribute-based Authenticated Key Exchange}\label{sec:abake}
Gorantla et al.\cite{gorantla2010attribute} introduced the concept of \gls{abake} using a attribute-based key encapsulation mechanism. In short this is a \gls{kem} with \gls{abe} as the encryption mechanism. The idea is that several users can exchange keys and thus communicate without knowing the identities of all the users. Any user satisfying the specified policy should be able to participate in the communication. \gls{abake} establish a common session key between the users which can be used to communicate securely. Goyal et al. \cite{ABE} introduced the notion of \gls{cpabe} where the private key of each user are associated with attributes and the cipher text has an attached access policy. The construction by Gorantla et al.\cite{gorantla2010attribute} uses this approach to create what they call an \gls{ep-ab-kem}, where the attributes are associated to the private key of a user and the access policy is attached to the encapsulation. The encapsulation is a randomly generated symmetric key encrypted with with a master public key and a access policy. To generate the common session key each user has to upload such an encapsulation and receive encapsulations from all other users. The session key is then obtained by decapsulating and combining the symmetric keys of all participants. 
\par \Gls{abake} will be described more in dept in the next chapter where a system implementing a variation of \gls{abake} is presented. 


\section{Applications}\label{sec:apps}
Key exchange schemes as discussed up till this point makes it feasible to exchange secret keys, and thus allow secure communication between users without them having to reveal their identities or to simply make it more feasible to use public key encryption. Identities may also be chosen differently depending on what domain or context the communication is being carried out in. The most general and intuitive case is simply using email or some other public identifier, but there may be cases where other identities is more useful. Within a company, working titles such as CEO or CTO could be used instead, to make it more usable in large companies where not everybody necessarily know the name of all their co-workers. In this case \gls{abake} would make it even more useful, since the CEO could have attributes including both his email and "CEO", allowing both identities to be used. 
\par Being able to communicate securely even without revealing identities is clearly useful for applications where users want to stay anonymous. Typically messaging services and forums can take advantage of such characteristics, users are able to exchange keys without any previous knowledge to eachother, while still knowing enough to trust them. The Internet is full of sites where users can upload questions which then can be answered by qualified superusers, but these services has the weakness that the users have to be willing to expose their message and possibly identity to be able to get an answer. After agreeing to this, they have to trust that the administrators of the system ensure the confidentiality of your message and only allow certain users to read it. The same goes for other applications where you want only specific types of users to be able to participate. 
\par By using functional key exchange you can specify in detail who is allowed to take part in the communication, this can range from very wide policies allowing a certain age group or gender, down to very specific characteristics such as degrees or military ranks. The most specific policy you can use is thus the identity itself as discussed earlier. This can be used in a variety of applications where access control of some degree is necessary, a good example being a room based chat system. There are several such applications where users are only allowed to join rooms if they satisfy some conditions, but usually the access control to the rooms rely on a server controlling this, so that when granted access, you have to trust that the service wont grant access to users without the correct attributes. Using functional key exchange, you as a user, would be able to ensure that nobody outside of the ongoing chat session will ever be able to read what is written. By the use of a session key relying on keying material from all participating parties. With this approach the system could inherit a hierarchy of user types, so that you have to prove your seriousness and knowledge to be able to achieve the higher rankings. This is a common way of administrating forums and chat room to avoid frivolous users who are there only to destroy the discussions. This can now be embedded directly in the encryption by adding group names as attributes in the key exchange policy. Functional key exchange can thus help prevent off-topic messaging and extraneous posting by users there with bad intentions.
In the next chapter a simple version of such a chat system will be presented to show how \gls{abake} can be used to keep the communication secure by the use of fresh session keys, renew every time a user joins. 
